{\footnotesize 
\begin{longtable}{p{.12\linewidth}p{.12\linewidth}p{.16\linewidth}p{.12\linewidth}p{.26\linewidth}p{.6\linewidth}}
Grid & Feature & Feature description & Variant & Variant description &  ID\\
Word order & Adpositions & Do adpositions normally occur before or after the noun? & Prep & Does the language have a substantial set of prepositions? E.g. English in the house & 213\\
Word order & Adpositions & Do adpositions normally occur before or after the noun? & Post & Does the language have a substantial set of postpositions? & 214\\
Word order & Noun-adjective & Do adjectives normally occur before or after the noun? & NA & Do most adjectives occur after the noun? & 215\\
Word order & Noun-adjective & Do adjectives normally occur before or after the noun? & AN & Do most adjectives occur before the noun? & 216\\
Word order & Noun-relative clause & Do relative clauses normally occur before or after the noun? & NRel & Do most relative clauses occur after the noun? & 217\\
Word order & Noun-relative clause & Do relative clauses normally occur before or after the noun? & RelN & Do most relative clauses occur before the noun? & 218\\
Word order & Noun-possessor & Do possessors normally occur before or after the noun? & N-Poss & Do most possessors occur after the noun they possess? The possessor should be an animate noun, and neither a proper name nor a pronoun! & 219\\
Word order & Noun-possessor & Do possessors normally occur before or after the noun? & Poss-N & Do most possessors occur before the noun they possess? The possessor should be an animate noun, and neither a proper name nor a pronoun! & 220\\
Word order & WH-element & What is the position of the WH-question word?  & WH-initial & Is the WH-question word always obligatorily the first element in a question (e.g., it does not trigger inversion)? & 221\\
Word order & WH-element & What is the position of the WH-question word?  & WH-V & Does the WH-question word always immediately precede the verb (i.e., stand directly before the verb, either initially or non-initially)? & 222\\
Word order & Main clause & What is the canonical (neutral) word order in a main clause? & SVO & What is the canonical (neutral) word order in a main clause? NB: V2 languages like Swe and Ger do NOT count as SVO even though SVO is most frequent. & 223\\
Word order & Main clause & What is the canonical (neutral) word order in a main clause? & V2 & V2 implies that initial adverb triggers V-SUBJ word order (Swedish, German etc.). & 224\\
Word order & Main clause & What is the canonical (neutral) word order in a main clause? & VSO & What is the canonical (neutral) word order in a main clause? & 225\\
Word order & Main clause & What is the canonical (neutral) word order in a main clause? & SOV & What is the canonical (neutral) word order in a main clause? & 226\\
Word order & Subordinate clause & What is the canonical (neutral) word order in a subordinate clause?  & SVO & What is the canonical (neutral) word order in a subordinate clause? NB: V2 languages like Swe and Ger do NOT count as SVO even though SVO is most frequent. & 227\\
Word order & Subordinate clause & What is the canonical (neutral) word order in a subordinate clause?  & V2 & V2 implies that initial adverb triggers V-SUBJ word order (Swedish, German etc.). & 228\\
Word order & Subordinate clause & What is the canonical (neutral) word order in a subordinate clause?  & VSO & What is the canonical (neutral) word order in a subordinate clause?  & 229\\
Word order & Subordinate clause & What is the canonical (neutral) word order in a subordinate clause?  & SOV & What is the canonical (neutral) word order in a subordinate clause?  & 230\\
Word order & Infinitive & Does the object normally occur before or after an infinitive? E.g.: to make pancakes (VO) Pfannkuchen machen (OV) & VO &  & 231\\
Word order & Infinitive & Does the object normally occur before or after an infinitive? E.g.: to make pancakes (VO) Pfannkuchen machen (OV) & OV &  & 232\\
Word order & Participle & Does the object normally occur before or after a participle?  E.g.: making pancakes (VO)  Pfannkuchen machend (OV) & VO &  & 233\\
Word order & Participle & Does the object normally occur before or after a participle?  E.g.: making pancakes (VO)  Pfannkuchen machend (OV) & OV &  & 234\\
Word order & Clitic pronouns finite verb & Does the clitic object pronoun normally occur before or after a finite verb?  E.g.: Je les fais. (OV)  If a language does not have clitic object pronouns, it would be 0 in both OV and VO. & VO &  & 235\\
Word order & Clitic pronouns finite verb & Does the clitic object pronoun normally occur before or after a finite verb?  E.g.: Je les fais. (OV)  If a language does not have clitic object pronouns, it would be 0 in both OV and VO. & OV &  & 236\\
Word order & Clitic pronouns finite verb & Does the clitic object pronoun normally occur before or after a finite verb?  E.g.: Je les fais. (OV)  If a language does not have clitic object pronouns, it would be 0 in both OV and VO. & 2nd position & Does the clitic pronoun always occur in 2nd position, not specifically before or after the verb? (Wackernagel position) & 237\\
Word order & Clitic pronouns infinitive & Does the clitic object pronoun normally occur before or after an infinitive? E.g.: Je veux les faire. (OV) If a language does not have clitic object pronouns, it would be 0 in both OV and VO. & VO &  & 238\\
Word order & Clitic pronouns infinitive & Does the clitic object pronoun normally occur before or after an infinitive? E.g.: Je veux les faire. (OV) If a language does not have clitic object pronouns, it would be 0 in both OV and VO. & OV &  & 239\\
Word order & Clitic pronouns infinitive & Does the clitic object pronoun normally occur before or after an infinitive? E.g.: Je veux les faire. (OV) If a language does not have clitic object pronouns, it would be 0 in both OV and VO. & 2nd position & Does the clitic pronoun always occur in 2nd position, not specifically before or after the verb? & 240\\
Word order & Clitic pronouns participle & Does the clitic object pronoun normally occur before or after a participle? E.g.: En les faisant... (OV) If a language does not have clitic object pronouns, it would be 0 in both OV and VO. & VO &  & 241\\
Word order & Clitic pronouns participle & Does the clitic object pronoun normally occur before or after a participle? E.g.: En les faisant... (OV) If a language does not have clitic object pronouns, it would be 0 in both OV and VO. & OV &  & 242\\
Word order & Clitic pronouns participle & Does the clitic object pronoun normally occur before or after a participle? E.g.: En les faisant... (OV) If a language does not have clitic object pronouns, it would be 0 in both OV and VO. & 2nd position & Does the clitic pronoun always occur in 2nd position, not specifically before or after the verb? & 243\\
Nominal morphology & Nominal case & What is the realization of case at nouns (nominal heads)? & O-case & Are there different noun forms for agent and object case? (English: 0 (no cases)  Russian: 1 (different noun forms for accusative and nominative)  Basque: 1 (different noun forms for ergative and absolutive) & 244\\
Nominal morphology & Nominal case & What is the realization of case at nouns (nominal heads)? & DAT & Is there a specific case form for the recipient, which is different from the case form of, e.g., the object?  (E.g. The man gives a book (O) to the child (DAT)) & 245\\
Nominal morphology & Nominal case & What is the realization of case at nouns (nominal heads)? & GEN & Is there a special case form to express genitive, which is different from the agent/object case? & 246\\
Nominal morphology & Nominal case & What is the realization of case at nouns (nominal heads)? & GEN/DAT & Is there a special noun form to express genitive, which is not the same as dative (recipient) case? & 247\\
Nominal morphology & Nominal case & What is the realization of case at nouns (nominal heads)? & VOC & Is there a special noun form to express vocative which is not the same as agent or object case? & 248\\
Nominal morphology & Nominal case & What is the realization of case at nouns (nominal heads)? & OBL-Cases & Are there any cases besides agent, object, genitive, dative, and vocative? (E.g., local cases) & 249\\
Nominal morphology & Nominal case & What is the realization of case at nouns (nominal heads)? & $>$7 Cases & Are there more than 7 cases? & 250\\
Nominal morphology & Nominal case & What is the realization of case at nouns (nominal heads)? & AGGL.CASE & Are there cases which are visibly agglutinative, i.e., built up by several distinct, segmentable affixes? & 251\\
Nominal morphology & Nominal case & What is the realization of case at nouns (nominal heads)? & AGGL.CASE.NR & Are plural cases formed by combining an (infixed) plural affix and a case affix in an agglutinative manner? & 252\\
Nominal morphology & Pronominal case & What is the realization of case at pronouns (pronominal heads)? Mainly 1st and 2nd person pronouns, ignoring 3rd person pronouns (which often come from demonstratives). & A $\neq$�O & In pronouns, is the marking different for the case of the agent and object?  & 253\\
Nominal morphology & Pronominal case & What is the realization of case at pronouns (pronominal heads)? Mainly 1st and 2nd person pronouns, ignoring 3rd person pronouns (which often come from demonstratives). & DAT $\neq$�O & In pronouns, is the marking different for the case of the recipient and the object?  & 254\\
Nominal morphology & Pronominal case & What is the realization of case at pronouns (pronominal heads)? Mainly 1st and 2nd person pronouns, ignoring 3rd person pronouns (which often come from demonstratives). & VOC & See Nominal case & 255\\
Nominal morphology & Pronominal case & What is the realization of case at pronouns (pronominal heads)? Mainly 1st and 2nd person pronouns, ignoring 3rd person pronouns (which often come from demonstratives). & OBL-Cases & See Nominal case & 256\\
Nominal morphology & Pronominal case & What is the realization of case at pronouns (pronominal heads)? Mainly 1st and 2nd person pronouns, ignoring 3rd person pronouns (which often come from demonstratives). & $>$7 Cases & See Nominal case & 257\\
Nominal morphology & Pronominal case & What is the realization of case at pronouns (pronominal heads)? Mainly 1st and 2nd person pronouns, ignoring 3rd person pronouns (which often come from demonstratives). & AGGL.CASE & See Nominal case & 258\\
Nominal morphology & Pronominal case & What is the realization of case at pronouns (pronominal heads)? Mainly 1st and 2nd person pronouns, ignoring 3rd person pronouns (which often come from demonstratives). & AGGL.CASE.NR & See Nominal case & 259\\
Nominal morphology & Case marking & On which elements of the NP is the case marking obligatory? & CASE-LAST & Is the case marking obligatory on the last element of the NP (i.e., it is only realized once in the NP, even if it consists of several elements)? & 260\\
Nominal morphology & Case marking & On which elements of the NP is the case marking obligatory? & CASE-FIRST & Is the case marking obligatory on the first element of the NP (i.e., it is only realized once in the NP, even if it consists of several elements)? & 261\\
Nominal morphology & Case marking & On which elements of the NP is the case marking obligatory? & CASE-N & Is the case marking obligatory realized on the noun? & 262\\
Nominal morphology & Case marking & On which elements of the NP is the case marking obligatory? & CASE-ADJ & Is the case marking obligatory on the adjective? & 263\\
Nominal morphology & Case marking & On which elements of the NP is the case marking obligatory? & CASE-ART & Is the case marking realized on the article? & 264\\
Nominal morphology & Gender / noun class & How is gender / noun class realized in the language? & M/F & Is there an obligatory gender distinction between masculine and feminine realized on an agreeing article or adjective?  Can be either on the adjective (Russian) or on both the article and the adjective (German), or even on a verb (as in some NE Caucasian languages).  & 265\\
Nominal morphology & Gender / noun class & How is gender / noun class realized in the language? & NEUTR & Is there a special neutral gender for nouns realized on an agreeing article,  adjective or verb? & 266\\
Nominal morphology & Gender / noun class & How is gender / noun class realized in the language? & ANIM & Is there a special noun class for non-human animates realized on an agreeing article, adjective or verb? & 267\\
Nominal morphology & Gender / noun class & How is gender / noun class realized in the language? & $<$5 GENDER & Are there more than 5 noun classes (or genders)? & 268\\
Nominal morphology & Definiteness marking & How is definiteness marking realized in the language? & DEF-ART & Is there a special word class of definite articles which occur in NPs without adjectives?  (E.g.: German, English but not Swedish) & 269\\
Nominal morphology & Definiteness marking & How is definiteness marking realized in the language? & N-DEF & Is there a suffix for definiteness on the noun? (E.g.: Swedish but not English!) & 270\\
Nominal morphology & Definiteness marking & How is definiteness marking realized in the language? & ADJ-DEF & Is there a suffix for definiteness on the adjective? This includes cases when the ADJ has a different form in definite and indefinite NPs (Swe ``det stora huset", Ger ``das grosse Haus"). & 271\\
Nominal morphology & Definiteness marking & How is definiteness marking realized in the language? & DEF-LAST & Is the definiteness marking obligatory on the last element of the NP (so it is only realized once in the NP, even if it consists of several elements)?  If there is no definiteness marking at all, it will be 0! & 272\\
Nominal morphology & Definiteness marking & How is definiteness marking realized in the language? & DEF-FIRST & Is the definiteness marking obligatory on the first element of the NP (so it is only realized once in the NP, even if it consists of several elements)?  If there is no definiteness marking at all, it will be 0! & 273\\
Nominal morphology & Gender agreement & How is gender agreement realized in the language? & PRED-ADJ & Does a predicative adjective agree with the subject of the clause in gender? & 274\\
Nominal morphology & Preposition agreement & How is preposition agreement realized in the language? & PREP-PRON-AGR & Can a preposition agree in person with its object? & 275\\
Verbal morphology & Simple PAST, A & In simple past, how is verbal agreement realized with respect to the agent? & PST:A-AGR-FULL & In simple past: does the verb crossreference the agent in all persons /numbers? & 276\\
Verbal morphology & Simple PAST, A & In simple past, how is verbal agreement realized with respect to the agent? & PST:NO-A-AGR & In simple past: does the verb not crossreference the agent on the verb at all (e.g., Swedish)? & 277\\
Verbal morphology & Simple PAST, A & In simple past, how is verbal agreement realized with respect to the agent? & PST:A-Gender-AGR & In simple past, does the verb agree in gender with the subject of a transitive verb? (e.g., Russian, Polish). & 278\\
Verbal morphology & Simple PAST, O & In simple past, how is verbal agreement realized with respect to the object?  & PST:O-AGR-FULL & In simple past: does the verb crossreference the object in all persons /numbers? & 279\\
Verbal morphology & Simple PAST, O & In simple past, how is verbal agreement realized with respect to the object?  & PST:NO-O-AGR & In simple past: does the verb not crossreference the object on the verb at all (e.g., Swedish, English, Russian)? & 280\\
Verbal morphology & Simple PAST, O & In simple past, how is verbal agreement realized with respect to the object?  & PST:O-Gender-AGR & In simple past, does the verb agree in gender with the object? & 281\\
Verbal morphology & Simple PAST, DAT & In simple past, how is verbal agreement realized with respect to the indirect object of ditransitive verbs? & PST:DAT-AGR-FULL & In simple past: does the verb crossreference the dative in all persons /numbers? & 282\\
Verbal morphology & Simple PAST, DAT & In simple past, how is verbal agreement realized with respect to the indirect object of ditransitive verbs? & PST:NO-DAT-AGR & In simple past: does the verb not crossreference the dative on the verb at all (e.g., Swedish, English, Russian) & 283\\
Verbal morphology & Simple PAST, DAT & In simple past, how is verbal agreement realized with respect to the indirect object of ditransitive verbs? & PST:DAT-Gender-AGR & In simple past, does the verb agree in gender with the indirect object of a ditransitive verb? & 284\\
Verbal morphology & Present progressive, A & In present progressive: how is verbal agreement realized with respect to the agent?  & PROG:A-AGR-FULL & In present progressive: does the verb crossreference the agent in all persons /numbers? & 285\\
Verbal morphology & Present progressive, A & In present progressive: how is verbal agreement realized with respect to the agent?  & PROG:NO-A-AGR & In present progressive: does the verb not crossreference the agent on the verb at all (e.g., Swedish) & 286\\
Verbal morphology & Present progressive, A & In present progressive: how is verbal agreement realized with respect to the agent?  & PROG:A-Gender-AGR & In present progressive, does the verb agree in gender with the subject of a transitive verb? & 287\\
Verbal morphology & Present progressive, O & In present progressive, how is verbal agreement organized with respect to the object? & PROG:O-AGR-FULL & In present progressive: does the verb crossreference the object in all persons /numbers? & 288\\
Verbal morphology & Present progressive, O & In present progressive, how is verbal agreement organized with respect to the object? & PROG:NO-O-AGR & In present progressive: does the verb not crossreference the object on the verb at all (e.g. Swedish, Russian)? & 289\\
Verbal morphology & Present progressive, O & In present progressive, how is verbal agreement organized with respect to the object? & PROG:O-Gender-AGR & In present progressive, does the verb agree in gender with the object of a transitive verb? & 290\\
Verbal morphology & Present progressive, DAT & In present progressive, how is verbal agreement organized with respect to the indirect object of ditransitive verbs? & PROG:DAT-AGR-FULL & Iin present progressive: does the verb crossreference the indirect object of a ditransitive verb in all persons /numbers? & 291\\
Verbal morphology & Present progressive, DAT & In present progressive, how is verbal agreement organized with respect to the indirect object of ditransitive verbs? & PROG:NO-DAT-AGR & In present progressive: does the verb not crossreference the indirect object of ditransitive verbs at all (e.g. Swedish, English, Russian)? & 292\\
Verbal morphology & Present progressive, DAT & In present progressive, how is verbal agreement organized with respect to the indirect object of ditransitive verbs? & PROG:DAT-Gender-AGR & In the present progressive, does the verb agree in gender with the indirect object of a ditransitive verb? & 293\\
Verbal morphology & Allocutive agreement & Does the verb agree with the receiver (the person one is speaking to) without the speaker being an argument in the sentence (allocutive agreement, probably no for all languages but Basque!)  & ALLOC & Does the verb agree with the receiver (the person one is speaking to) without the speaker being an argument in the sentence (allocutive agreement, probably no for all languages but Basque!) & 294\\
Tense & Future & How is future realized in the language? & FUT.AUX & Is there a future formed by an auxiliary? (E.g., will in English? & 295\\
Tense & Future & How is future realized in the language? & PERF.FUT & Is there a future formed by using the perfective aspect? (0 if the language does not have verbal aspects! E.g., Russian, Georgian) & 296\\
Tense & Future & How is future realized in the language? & FUT.Participle & Is there a future formed by a participle? (E.g., Armenian, Basque) & 297\\
Tense & Future & How is future realized in the language? & FUT.Particle & Is there a future formed by a particle preceding a finite verb? (E.g., Albanian) & 298\\
Tense & Future & How is future realized in the language? & FUT.Synth & Is there a synthetical future? (E.g., French, Spanish) & 299\\
Tense & Continous present & How is present progressive realized in the language? & Present & Is there a synthetic present in progressive function? & 300\\
Tense & Continous present & How is present progressive realized in the language? & Progressive present & Is there a progressive present form constructed by combining a present participle with a finite auxiliary verb? & 301\\
Alignment & Noun: Simple Past & In simple past: how is the marking of subject and object of nouns realized? & N:PST:A=O? & In simple past: Is the noun form for A the same as for O? Ie: Does the noun look the same when it is subject of a transitive clause than when it is object of a transitive clause? & 302\\
Alignment & Noun: Simple Past & In simple past: how is the marking of subject and object of nouns realized? & N:PST:A=Sa? & In simple past:Is the noun form for A the same as for Sa? Ie: Does the noun look the same when it is subject of a transitive clause as when it is subject of an agentive intransitive verb such as ``work" or ``dance"? & 303\\
Alignment & Noun: Simple Past & In simple past: how is the marking of subject and object of nouns realized? & N:PST: O=So? & In simple past: Is the noun form for O the same as for So? Ie: Does the noun look the same when it is object of a transitive clause as when it is subject of an unaccusative verb such as ``fall" or ``die"? & 304\\
Alignment & Noun: Simple Past & In simple past: how is the marking of subject and object of nouns realized? & N:PST: Sa=So? & In simple past: Does a noun bear the same case form when it is Sa (subject of e.g. work) or So (subject of e.g. fall or die)? Ie: There does not exist a split into stative and active intransitive verbs. & 305\\
Alignment & Noun: Present Progressive & In present progressive: how is the marking of subject and object of nouns realized?  & N:PROG: A=O? & In present progressive: Is the noun form for A the same as for O?  I.e.: Does the noun look the same when it is subject of a transitive clause and when it is object of a transitive clause? & 306\\
Alignment & Noun: Present Progressive & In present progressive: how is the marking of subject and object of nouns realized?  & N:PROG: A=Sa? & In present progressive:Is the noun form for A the same as for Sa?  I.e.: Does the noun look the same when it is subject of a transitive clause and when it is subject of an agentive intransitive verb such as ``work" or ``dance"? & 307\\
Alignment & Noun: Present Progressive & In present progressive: how is the marking of subject and object of nouns realized?  & N:PROG: O=So? & In present progressive: Is the noun form for O the same as for So?  I.e.: Does the noun look the same when it is object of a transitive clause and when it is subject of an unaccusative verb such as ``fall" or ``die"? & 308\\
Alignment & Noun: Present Progressive & In present progressive: how is the marking of subject and object of nouns realized?  & N:PROG: Sa=So? & In present progressive: does a noun bear the same case form when it is Sa (subject of e.g., ``work") or So (subject of e.g., ``fall" or ``die")?  I.e.: The language does not have a split between stative and active intransitive verbs. & 309\\
Alignment & Pronoun: Simple Past & In present progressive: how is the marking of subject and object of pronouns realized? & P:PST: A=O? & In simple past: Is the pronoun form for A the same as for O?  I.e.: Does the pronoun look the same when it is subject of a transitive clause than when it is object of a transitive clause? & 310\\
Alignment & Pronoun: Simple Past & In present progressive: how is the marking of subject and object of pronouns realized? & P:PST: A=Sa? & In simple past: Is the pronoun form for A the same as for Sa?  I.e.: Does the pronoun look the same when it is subject of a transitive clause than when it is subject of an agentive intransitive verb such as ``work" or ``dance"? & 311\\
Alignment & Pronoun: Simple Past & In present progressive: how is the marking of subject and object of pronouns realized? & P:PST: O=So? & In simple past: Is the pronoun form for O the same as for So?  I.e.: Does the pronoun look the same when it is object of a transitive clause than when it is subject of an unaccusative verb such as ``fall" or ``die"? & 312\\
Alignment & Pronoun: Simple Past & In present progressive: how is the marking of subject and object of pronouns realized? & P:PST: Sa=So? & In simple past: Does the pronoun bear the same case form when it is Sa (subject of e.g. work) or So (subject of e.g. fall or die)?  I.e.: There does not exist a split into stative and active intransitive verbs. & 313\\
Alignment & Pronoun: Present Progressive &  In present progressive: how is the marking of subject and object of pronouns realized? & P:PROG: A=O? & In present progressive: Is the pronoun form for A the same as for O?  I.e.: Does the pronoun look the same when it is subject of a transitive clause than when it is object of a transitive clause? & 314\\
Alignment & Pronoun: Present Progressive &  In present progressive: how is the marking of subject and object of pronouns realized? & P:PROG: A=Sa? & In present progressive:Is the pronoun form for A the same as for Sa?  I.e.: Does the pronoun look the same when it is subject of a transitive clause as when it is subject of an agentive intransitive verb such as ``work" or ``dance"? & 315\\
Alignment & Pronoun: Present Progressive &  In present progressive: how is the marking of subject and object of pronouns realized? & P:PROG: O=So? & In present progressive: Is the pronoun form for O the same as for So?  I.e.: Does the pronoun look the same when it is object of a transitive clause than when it is subject of an unaccusative verb such as ``die" or ``fall"? & 316\\
Alignment & Pronoun: Present Progressive &  In present progressive: how is the marking of subject and object of pronouns realized? & P:PROG: Sa=So? & In present progressive: Does a pronoun bear the same case form when it is Sa (subject of e.g. work) or So (subject of e.g. fall or die)? Ie: There does not exist a split into stative and active intransitive verbs. & 317\\
Alignment & Verb: Simple Past & In simple past, how is alignment realized on the verb? & V:PST:A=O? & In simple past: Is the verb affix for A the same as for O?  I.e.: Does the verb look the same when it refers to the  subject of a transitive clause than when it refers to the object of a transitive clause?  If there is no O-marking on the verb, but there is an S-marking, the answer would be no, they do not look the same. (e.g., German, Russian)  If there is neither an O, nor an A marking, like in Swedish, the answer would be yes, they look the same! & 318\\
Alignment & Verb: Simple Past & In simple past, how is alignment realized on the verb? & V:PST:A=Sa? & In simple past: Is the verb affix for A the same as for Sa?  I.e.: Does the verb look the same when it refers to subject of a transitive clause than when it refers to subject of an agentive intransitive verb like ``work" or ``dance"? & 319\\
Alignment & Verb: Simple Past & In simple past, how is alignment realized on the verb? & V:PST:O=So? & In simple past: Is the verb affix for O the same as for So?  I.e.: Does the verb look the same when it refers to the object of a transitive clause as when it refers to the subject of an unaccusative verb (such as ``fall" or ``die")? & 320\\
Alignment & Verb: Simple Past & In simple past, how is alignment realized on the verb? & V:PST:Sa=So? & In simple past: Is the verb affix the same for Sa (subject of e.g. work) as or So (subject of e.g. fall or die)? I.e., does the verb agreement affix look the same regardless of whether the verb is ``work" or ``die" (as in German: ``arbeitete-st", ``starb-st").  I.e.: There does not exist a split into unaccusative and agentive intransitive verbs. & 321\\
Alignment & Verb: Present Progressive & In present progressive, how is alignment realized on the verb? & V:PROG: A=O? & In present progressive: Is the verb affix for A the same as for O? Ie: Does the verb look the same when it refers to the  subject of a transitive clause than when it refers to the object of a transitive clause? If there is no O-marking on the verb, but there is an S-marking, the answer would be no, they do not look the same. (e.g. German, Russian) If there is neither an O, nor an A marking like in Swedish, the answer would be yes, they look the same! & 322\\
Alignment & Verb: Present Progressive & In present progressive, how is alignment realized on the verb? & V:PROG: A=Sa? & In present progressive:Is the verb affix for A the same as for Sa? Ie: Does the verb look the same when it refers to subject of a transitive clause as when it refers to subject of an agentive intransitive verb such as ``work"? & 323\\
Alignment & Verb: Present Progressive & In present progressive, how is alignment realized on the verb? & V:PROG: O=So? & In present progressive: Is the verb affix the same for O as for So? Ie: Does the verb look the same when it refers to the object of a transitive clause as when it refers to the subject of an unaccusative verb (such as ``fall" or ``die")? & 324\\
Alignment & Verb: Present Progressive & In present progressive, how is alignment realized on the verb? & V:PROG: Sa=So? & In present progressive: Is the verb affix the same for Sa (subject of e.g. ``work") as or So (subject of e.g. ``fall" or ``die")? I.e. does the verb agreement affix look the same regardless of whether the verb is ``work" or ``die" (as in German: arbeite-t, stirb-t).  I.e.: There does not exist a split into unaccusative and agentive intransitive verbs. & 325\\
Alignment & Compare PROG-PAST & What is the marking relation between subject and object in present progressive and simple past? & PROG\underline{\phantom{X}}So= PAST\underline{\phantom{X}}So & Does the subject of e.g. die or fall bear the same case in both progressive present and simple past? (the answer for e.g., Megrelian would be no) & 326\\
Alignment & Compare PROG-PAST & What is the marking relation between subject and object in present progressive and simple past? & PROG\underline{\phantom{X}}A= PAST\underline{\phantom{X}}O & Does the subject of a transitive verb in the present progressive bear the same case form as the object of a transitive verb in the simple past? (e.g. as in Georgian) & 327\\
Alignment & Compare PROG-PAST & What is the marking relation between subject and object in present progressive and simple past? & PAST\underline{\phantom{X}}A= PROG\underline{\phantom{X}}O & Does the subject of a transitive verb in the simple past bear the same case form as the object of a verb in the present progressive? (e.g., Kurdish) & 328\\
Alignment & Reflexive pronoun in transitive clause & What is the alignment of reflexive pronouns? & REFL-ref-A & In a transitive clause, can O be a reflexive which refers back to A (as in English ``herself", Swedish ``sig")? & 329\\
Alignment & Reflexive pronoun in transitive clause & What is the alignment of reflexive pronouns? & REFL-ref-O & In a transitive clause, can A be a reflexive which refers back to O (as appears to be the case in some Caucasian languages)? & 330
\end{longtable}
}